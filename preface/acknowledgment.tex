\prefacesection{Acknowledgments}
In terms of both scholarly and personal support, this dissertation would not have been possible without the help of many people. I would like to thank my thesis committee members, Professors Mykel Kochenderfer, Stephen Boyd, and Charbel Farhat, for providing many interesting discussions and taking the time to critique this work. I would also like to thank the researchers and staff of the Jicamarca Radio Observatory, particularly Dr. Jorge Chau and Karim Kuyeng, for hosting me and running my experiments. At Stanford, the Space Environment and Satellite Systems lab has been my home for most of my graduate studies, and I would like to thank its past and present members for making that home a good one. Thanks must also go to my classmates and friends in the Aero/Astro department, particularly my study group, Tom Coles, Brandon Fetroe, and Jason Forshaw, and my quals group, Brandon Morgan and Surajit Roy.

My first crash course in radar came on a two-week visit to the MIT Haystack Observatory to collect data and attend the AMISR workshop. I knew almost nothing about collecting and processing radar data at that point in time, but after those two weeks I knew enough to be dangerous. Credit and thanks go to the teachers at the workshop and the researchers at Haystack, most notably my co-advisor of sorts, Dr. Phil Erickson. On the first day of that visit, Phil sat me down in his office, gave my some Python code (with me never having used the language before), and let me go at it. He has answered my questions ever since, providing insight whenever I've been stuck on the thorny details of radar. I would like to thank Phil for being an outstanding teacher and mentor.

I was fortunate enough to be at the right place at the right time when I met my advisor, Professor Sigrid Close. Again, I knew nothing about radar or the ionosphere before that first meeting almost five years ago, and I had been beginning to wonder whether I would be able to find a good fit for my interests at Stanford. Sigrid solved those problems by providing me with ALTAIR meteor data to analyze and giving me the freedom to explore. She was enthusiastic and supportive when I later came to her office with ambitious ideas about applying compressed sensing to radar, even if it meant that I never actually got around to analyzing that ALTAIR data. I would like to thank Sigrid for providing many work and travel opportunities and for guiding my scholarly explorations with the perfect blend of support and advice.

Finally, I would like to thank my friends and family for their encouragement, especially for not asking too many questions about when this thesis would finally be done. Most importantly, though, I want to thank my wife, Bekah, without whom none of this would have happened. It has been a long journey to get to this point, and I'm glad I could share the highs and lows with her.

\begin{flushright}
 \textit{Ryan Volz}\\
 \textit{Stanford, California}\\
 \textit{December, 2014}
\end{flushright}

\vfill
\noindent
\begin{minipage}{\textwidth}
 \footnotesize%
 This research was supported by the Department of Defense (DoD) through the National Defense Science \& Engineering Graduate Fellowship (NDSEG) Program, and by the National Science Foundation under Grant No. AGS-1056042.
\end{minipage}