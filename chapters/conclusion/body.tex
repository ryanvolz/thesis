\chapter{Conclusion}
\label{conclusion}
With the goal of enabling flexible, high-resolution measurements of ionospheric plasma phenomena, this thesis draws upon the subjects of radar analysis, sparse approximation, and convex optimization to develop and apply a novel waveform inversion method. In Part \ref{part_introduction}, we introduced these subjects and described the challenges encountered with ionospheric radar measurements that arise from the inherent conflict between sensitivity and ambiguity. Current techniques struggle to meet the following goals:
\begin{inparaenum}[1)]
 \item Differentiation of targets in crowded and variable environments, which is necessary to separate meteor and electrojet signals in the equatorial ionosphere;
 \item Elimination of self-interference of range- or frequency-spread targets, including non-specular meteor trails;
 \item High range and frequency resolution for observing small-scale processes, such as meteoroid fragmentation and meteor flares; and
 \item Flexibility to achieve these goals for a variety of waveform and measurement requirements arising from experimental and hardware constraints.
\end{inparaenum}
Waveform inversion can achieve these goals. In Part \ref{part_radar_analysis}, we reviewed the traditional radar matched filtering technique and the ambiguity that it introduces, and we proposed a mathematical model to describe radar measurements in terms of ambiguity-free reflectivity coefficients. In Part \ref{part_waveform_inversion}, we addressed the complete problem of waveform inversion, from the theory of compressed sensing via convex optimization to the implementation of efficient algorithms and the application to actual radar data. In this chapter, we conclude with a discussion of contributions and a sketch of potential further research.

\section{Summary and Discussion of Contributions}
\label{discussion}
\subsection{Radar Model}
\begin{italicquote}
A new perspective for radar measurements is espoused in which the radar scene reflectivity is recovered by inverting a mathematical model.
\end{italicquote}

Radar measurements in their basic form, a sequence of values sampled from the received signal, are not amenable to information processing when pulses longer than the sampling period are used. Our radar model addresses this by representing the measurements as a function of delay-frequency reflectivity coefficients. The use of a delay-frequency dictionary is useful because the coefficients are typically sparse. A similar delay-frequency analysis is also possible with a bank of frequency-shifted matched filters, but that representation is blurred with respect to the true reflectivities due to waveform ambiguity. We first introduced the radar model in Chapter \ref{radar_model} with exactly this imaging analogy; the ambiguity function acts as a point-spread function when matched filtering is applied to the measurements, blurring the true radar scene's reflectivity. The forward radar model is the adjoint of the matched filter bank operation, and as a consequence, matched filtering gives the least-norm solution to the radar model measurement equation. Instead, knowing that a sparse solution is most likely to be correct, we posed the following waveform inversion problem: invert the radar model with sparsity as a constraint to recover the sidelobe-free radar scene.

\begin{italicquote}
 A discrete radar model is rigorously derived from the continuous measurement equation for a general target scene.
\end{italicquote}

We followed the informal radar model presentation with a rigorous derivation to determine precisely how general radar scenes are represented. Starting from the continuous narrow-band radar measurement equation, the discrete radar model was re-derived through discretization and Fourier analysis. Notably, this derivation only assumes what is typical for matched filter analysis: a narrow-band signal, no scattering from outside the delay window of interest, uniform sampling of the received signal, a transmission waveform represented by a code sequence describing constant bauds, and a number of frequency steps greater than the length of the code. The resulting radar model is discrete, with a fixed number of reflectivity coefficients corresponding to a delay-frequency grid. Point targets falling on the grid points are naturally represented, but off-grid point targets and distributed targets are also accommodated. The derivation shows that arbitrary targets are written exactly, though not uniquely, in terms of the gridded coefficients. With respect to delay, the coefficients represent the integrated reflectivity over the corresponding delay window; with respect to frequency, the coefficients represent samples from the complete reflectivity function after it has been convolved with a periodic sinc function. Because of this known relationship, we can infer that sparsity of the full radar scene, as represented by the continuous reflectivity function, is reasonably preserved as sparsity of the discrete reflectivity coefficients. Thus, the waveform inversion procedure is justified.

\subsection{Waveform Inversion Method}
\begin{italicquote}
 A novel waveform inversion technique is formulated based on finding a sparse representation of the radar scene using the discrete model.
\end{italicquote}

Sparse solutions to underdetermined systems of equations is the domain of the theory of compressed sensing, which states that minimizing the $l_1$-norm when constrained by enough incoherent measurements guarantees recovery of the true sparse solution. In Chapter \ref{waveform_inversion}, a review of theoretical results indicated that recovery of the reflectivity coefficients using the radar model is likely to be successful provided that a suitable waveform is used. All that is required are global, incoherent measurements, which are ensured by random binary-phase sequences, the Alltop quadratic chirp sequences, and likely many more waveforms. With successful recovery, waveform inversion meets all of the goals that existing methods fail to reach. Elimination of delay-frequency sidelobes, and their ambiguity, is inherent to the method. That, coupled with the simultaneous decoding over the signal's full frequency spectrum provided by waveform inversion, enables target differentiation for multiple or distributed scatterers in crowded and variable environments. Given a sufficient level of sparsity, inversion can also provide solutions with a higher resolution than that specified by the sampling rate, thus enabling better observations of small-scale processes. Since recovery is likely with many different waveforms, the method is also flexible enough to be used when there is little or no freedom in choosing the waveform or dictating the experimental setup.

In practical terms, the waveform inversion solution procedure is straightforward. Our approach, detailed in Chapter \ref{waveform_inversion}, is to solve an $l_1$-regularized least squares problem using a first-order convex optimization method that involves the soft-thresholding proximal operator. Algorithms like this have an intuitive interpretation in radar terms as an iterative thresholding matched filter: each step involves applying the matched filter to detect under-represented components and thresholding the new representation to induce sparsity. The optimization problem's regularization parameter $\lambda$ is chosen to ensure a constant false-alarm rate for detection based on an independently-estimated noise level. In order to compare the inversion to matched filtering and de-bias the result, the solution is scaled and added to its matched filtered measurement residual. Adding the residual term also has the benefit of including any signal that falls below the detection threshold. As a result, the regularization parameter (detection threshold) serves as the threshold for sidelobe elimination: sidelobes are removed for signals with magnitude above the threshold and left intact for signals with magnitude below the threshold. Larger thresholds result in faster convergence, so there is a direct trade-off between speed and sidelobe elimination.

\begin{italicquote}
 The waveform inversion method is implemented using modern convex optimization techniques tailored for efficient computation.
\end{italicquote}

Our waveform inversion method was implemented as a pair of Python packages, \pkg{radar\-model} and \pkg{prx}, that provide the radar model operators and the proximal convex optimization algorithms. The forward and adjoint operations of the radar model are applied repeatedly in iterating to the solution, so they must be as efficient as possible. Our implementation uses the FFT (fast Fourier transform) algorithm and multiple formulations of each operator to minimize computation time for arbitrary model dimensions. To ensure quick convergence to the solution, the convex optimization algorithms that we implemented include adaptive acceleration and step size schemes, including one of our own devising in the case of the linearized ADMM (alternating direction method of multipliers) algorithm.

We tested multiple first-order convex optimization algorithms that use the proximal operator to show that the accelerated proximal gradient method applied to $l_1$-regularized least squares converges the most quickly, in general, for the radar model. The most important point concerning these implementations, however, is that they are flexible and easy to understand. Algorithm development for problems of this type is advancing rapidly, and it is reasonable to expect faster, better algorithms to emerge before long, if they do not exist already. The \pkg{radarmodel} package can be used with any future algorithm that permits direct computation of the model operators, while the proximal methods found in \pkg{prx} are simple enough to easily accommodate improvements or, if completely obsoleted, serve as an intuitive explanation of the inversion process.

\subsection{Experimental Results}
\begin{italicquote}
 The real-world flexibility and effectiveness of the inversion technique is demonstrated by the elimination of filtering artifacts from meteor observations made with a variety of standard radar waveforms.
\end{italicquote}

Compressed sensing theory indicates that the waveform inversion technique is well-posed, but the more pertinent question concerns its practical application to analyzing ionospheric radar data. As proof of its effectiveness, we performed waveform inversion on meteor measurements collected using the Jicamarca 50 MHz radar. The results were dissected pulse by pulse, analyzed as individual frequency slices, and integrated over all frequencies to show that matched filter sidelobes are effectively eliminated with minimal remaining artifacts. Five common waveforms were compared in the experiment by transmitting them in a repeating sequence, one per pulse, so that they could be observed scattering from the same targets. Based on this sample, in which inversion was a moderate success using an uncoded pulse and a complete failure using a linear frequency-modulated pulse, we concluded that successful inversion depends on the compactness of the peak of the waveform's ambiguity function. Comparing the inversions to each other revealed that waveform-dependent artifacts remaining in the solutions were likely due to mismatch between the transmitted waveform and its idealized model. Additional inversions were attempted with undersampled data, producing mixed results that are promising for increased-resolution recovery but require further testing to verify. For future applications, we concluded that long, coded waveforms with good fidelity should be the goal, in order to achieve the best sensitivity and delay-frequency resolution with minimal mismatch artifacts.

\section{Future Work}
\label{future_work}
In this dissertation, we have built a strong foundation for the use of sparsity to achieve waveform inversion of radar measurements. Based on demonstrated and theoretical potential, this foundation warrants further development so that its promise is realized. There are many avenues for future research based on this work. Some involve specific applications of the waveform inversion technique, while others would strengthen the theory or develop additional methods. The following topics are illustrative of all of these possibilities.

\subsection{Incoherence and the Ambiguity Function}
Our experimental results have hinted at a strong connection between the radar ambiguity function and the compressed sensing concept of measurement incoherence. Indeed, they both describe the degree to which individual measurements are related. With a little effort, a direct relationship giving the incoherence in terms of the ambiguity function should be easy to establish. One consequence would be the ability to prove that certain waveforms are optimal for compressed sensing. The more interesting aspect of this concept, however, comes from the fact that both topics are individually well-studied. What known results concerning waveform ambiguity can be applied to incoherence, and vice-versa? Perhaps codes that are well-known to the radar community would be very useful in other fields of compressed sensing, and maybe the reverse is true as well. It is certainly possible to continue to apply the waveform inversion technique without further exploring this relationship, but we believe applications would benefit from more theoretical background in this area.

\subsection{Super-resolution}
The experimental results concerning double resolution reconstruction are enticing in their supposed quality, but we have not produced enough evidence as of yet to verify them. Even beyond doubling the resolution, sparse recovery has the potential to increase resolution right up to the undersampling limit prescribed by the recovery phase transition. A proper verification of these capabilities would involve deliberately discarding data when performing the reconstruction, much as we did for our half-sample results. Where we believe our results failed, and where future experiments must succeed, is in controlling for the effect of transmitted waveform fidelity. Incorporating the effects of lowpass filtering seemed to improve our double-resolution results compared to the half-sample results, and it makes sense that correctly matching the transmitted waveform with the model would be more important when working with less data. If super-resolution can be demonstrated and perfected, waveform inversion would be unique in reliably and automatically providing that capability.

\subsection{Multi-pulse Analysis}
Another avenue of research encouraged by the experimental results is the ability to combine data from multiple pulses to improve recovery. Provided that the radar scene maintains the same delay-frequency reflectivity over the length of multiple pulses, the combined data can be used to significantly increase the recovered frequency resolution. Even if some of the target properties change from pulse to pulse, the data is still likely to be correlated in some way. In a sense, this makes the scene even more sparse; instead of recovering each pulse individually, we can use correlation from pulse to pulse as a type of time sparsity to recover the reflectivities for multiple pulses using fewer measurements. Many formulations of this type of recovery are possible. One simple method would be to weight the individual coefficients in the $l_1$-norm penalty by the inverse of the previous pulse's solution, placing a higher penalty on coefficients expected to be zero. A statistical approach could involve a Kalman or particle filter to model the state of the radar scene and include measurement uncertainty; each pulse's sparse estimate could then be recovered while incorporating and updating the estimated statistical distribution. While the benefits of multi-pulse analysis are enticing, implementations will have to be careful to not significantly increase the dimensions of the model and thereby make recovery practically impossible.

\subsection{Alternative Radar Models}
The grid-of-point-targets radar model is successful and attractive because of its simplicity and direct relationship to established techniques, but there are aspects of it that can be improved. One troublesome feature is the blurring of the radar scene's frequency spectrum in the reflectivity coefficients by the periodic sinc function. Though it doesn't completely eliminate sparsity, it does increase the number of non-zero coefficients. Another troublesome aspect of the current radar model is its frequency coherence. In order to increase frequency resolution, it is necessary to increase the coherence of the measurements. The radar model is functionally equivalent to time-frequency analysis using the short-time Fourier transform, and their shortcomings are the same. Windowing is often used in Fourier analysis to minimize frequency spreading from discretization, so a windowed radar model could be an improvement. Wavelets are often a better method for analyzing signals that are compact in time and frequency, so a wavelet-based radar model would provide a better representation of the expected sparsity. Other techniques from the field of time-varying signal processing could surely be adapted as well. The key to implementing and actually using these enhancements in practice would be to ensure that they do not significantly increase the already-restrictive computational complexity of the model.

\subsection{Radar Self-calibration}
Waveform fidelity is an important issue for minimizing mismatch errors in the recovered solution, and the best way to know the transmitted waveform is to measure it directly. When this is not possible, however, a type of radar self-calibration could suffice. Consider the typical situation, in which we use an assumed ideal waveform to recover a sparse solution. The result includes the signal that we want, but it can also have mismatch artifacts that manifest as seemingly-random non-signal components in the sparse solution. In other words, we know that the true solution should actually be even more sparse. Ideally, we would be able to adjust the modeled transmitted waveform so that the recovered solution reaches the true level of sparsity and is free of artifacts. Self-calibration would do this by flipping the waveform inversion problem on its head, instead using the recovered reflectivity to solve for the transmitted waveform from the measurements. Perhaps, instead, it simultaneously recovers both the transmitted waveform and the reflectivity coefficients. Either way, we would apply knowledge about the form of transmitted waveform, along with knowledge that the reflectivity coefficients should be sparse, to constrain the solution. Since it is completely ambiguous whether a specific effect is attributed to the transmitted waveform or the reflectivity coefficients, applying the appropriate constraints would be the difficult part. If it could be made to work reliably, however, the quality of the solution would be unsurpassed.

\subsection{Passive Radar}
Passive radar is the practice of intercepting existing radio transmissions to make radar measurements \autocite{LEC+13}. One notable existing transmission is digital television, and with their large coverage and wide bandwidth, digital television signals make a very attractive target for passive radar. Since we obviously cannot control the waveform of the intercepted signal, it is not possible to use codes with good sidelobe properties to make the measurements. Consequently, delay-frequency sidelobes can pose a problem to passive radar accuracy, and waveform inversion provides an obvious improvement. The principal problem with direct application of the technique is that passive radar already requires significant processing power when using just a matched filter. Adding hundreds of iterations to perform waveform inversion would make the computation time unmanageable. Though the application to passive radar would appear to be straightforward, computational efficiency is the problem that would have to be solved to enable success.

\subsection{Near-universal Measurement Modes}
One difficulty with current ionospheric measurements is that the experiment has to be tailored to the specific phenomena being measured. The varied scattering processes cover a very wide range of both spatial and temporal coherence as a function of altitude, and these properties impose constraints on the pulse length and inter-pulse period needed to ensure unambiguous sampling. The requirements are often at odds for different processes, preventing simultaneous measurement with the same radar. For instance, ionospheric D-region scattering is relatively compact in altitude extent but has a long correlation time, requiring long sequences of short pulses to measure its autocorrelation function. Conversely, F-region scattering has a wide altitude extent but a short correlation time, which requires oversampling a long pulse to measure the autocorrelation well. Delay-frequency sidelobes prevent measuring the D-region with a long pulse, so these requirements are at odds. Waveform inversion, however, eliminates sidelobes and makes simultaneous measurement possible. In fact, in order to properly sample any autocorrelation function over a wide range of altitudes under a transmitter duty cycle constraint, it is necessary to use aperiodic waveform sequences \autocite{VVL09}. Matched filtering is wholly inadequate for decoding such sequences, but waveform inversion can decode them and enable a near-universal measurement mode for the entire ionosphere.